\documentclass[12pt]{article}
\title{AI-powered CYOA dev diary}
\begin{document}



\begin{itemize}
\item[May 13th 2025] Working on the database setup. Realized that
  currently I'm learning how to do the absolute most basic-bitch CRUD
  setup possible, just figuring out how to run Cloud SQL and get a
  local dev environment going for testing, so I should not worry at
  all about database design. For example, can you reuse Locations in
  Stories, or does each Location have a unique Story it's contained
  within? Doesn't matter at all for my purposes today, which are just
  that I can create and read back some dang text from the
  database. So, first milestone: Enable creating new locations,
  listing them, and displaying them on the web UI. They won't have any
  sort of relationships with other objects because today I'm not doing
  DB design, I'm doing DB Hello World.
\item[May 17th] Much struggle setting up protoc and whatnot, since I'm
  not using Bazel any more as I'm used to. Very annoying. Now need to
  figure out how to integrate that in the server.
\item[June 5th] Have completed very basic CRUD setup for
  Locations. Used Jules to produce the boilerplate for the Update and
  Delete operations, not sure it saved me much time but oh well, it
  was an experiment. Next step is to make all this compile on Cloud,
  since I now have two steps, the proto and then the server.
\item[June 13th] Finally completed migration setup for Cloud SQL using
  \verb|migrate.go| script. Gemini was helpful with the basic
  framework but extremely sloppy about getting the strings correct: It
  tried to set me up with a \verb|cloudsqlconn| custom dialer but gave
  me a \verb|unix| DSN. It insisted that the user for IAM
  authentication must be the full email address,
  \verb|runner@whatever.com|, but did not distinguish that from the
  database user for MySQL, which has a length limit of 32 characters
  and was consequently just \verb|runner|. Very much not oneshot
  success here, although on the other hand I'm not sure I would have
  succeeded at all without getting that basic framework, so. Room for
  improvement but still helpful, really.
\item[June 14th] And finally I have it! Refactoring \verb|main.go| to
  use the new initialization was easy enough, but I struggled all day
  with making it authorize correctly to Cloud SQL; it would chuck me
  out every time with ``access denied (using password: YES)''. Well I
  was not using any dang password, and spent literally hours
  confirming that every way I could think of! Finally thought to check
  the CloudSQL-side logs and found that the user was my service runner
  account right enough, but the authentication was still the default
  service account. Apparently gcloud doesn't respect the service
  account set in the trigger! Fixed by adding
\begin{verbatim}
--service-account=cyoa-dev-mysql-runner@${_AR_PROJECT_ID}.iam.gserviceaccount.com
\end{verbatim}
  to the YAML build step. Victory! CRUD now running in the Cloud!
\item[June 16th] So next step is to create Stories, which presumably
  consist of a starting Location and a map from Locations to Actions,
  which may have effects within the Location or may cause a transition
  to a new Location. And having gotten that, I will want to create
  Playthroughs, which reminds me that Stories need to have Characters
  which can be precreated or custom. But first the Actions.
\item[June 25th] I have been plugging away at actions. Copied my
  predicate library out of Landn�m, which saves some work. The Landn�m
  version is on Bazel BUILD files and I cannot be bothered with
  figuring out how to combine that with regular Go build system, so
  copying it is, sigh. Then wrote an E2E test which creates a trivial story,
  two locations and two possible actions in each location, and plays
  through the four possible combinations. When that has four endings
  and passes, well! It'll be frontend time!
\item[June 27th] Ah hah I just realised the problem here. I have a
  backend which handles game mechanics; I want to add a separate,
  pluggable service which adds narrative. So, what if the actions pass
  through the mechanical backend but fail out on the narrative? Could
  in principle gate the transaction commit on it, but seems
  fraught. Think the solution must be: Separate committing ``the
  player did this'' from committing ``these were the effects''. Hum!

  No actually this is not so difficult, just separate out read-only
  loading of state, hypothetical application of action and getting of
  narrative, and actual writing back to DB into separate bits. That
  way you're never left in an inconsistent state.
\end{itemize}

