\documentclass[12pt]{article}
\title{AI-powered CYOA dev diary}
\begin{document}



\begin{itemize}
\item[May 13th 2025] Working on the database setup. Realized that
  currently I'm learning how to do the absolute most basic-bitch CRUD
  setup possible, just figuring out how to run Cloud SQL and get a
  local dev environment going for testing, so I should not worry at
  all about database design. For example, can you reuse Locations in
  Stories, or does each Location have a unique Story it's contained
  within? Doesn't matter at all for my purposes today, which are just
  that I can create and read back some dang text from the
  database. So, first milestone: Enable creating new locations,
  listing them, and displaying them on the web UI. They won't have any
  sort of relationships with other objects because today I'm not doing
  DB design, I'm doing DB Hello World.
\item[May 17th] Much struggle setting up protoc and whatnot, since I'm
  not using Bazel any more as I'm used to. Very annoying. Now need to
  figure out how to integrate that in the server.
\item[June 5th] Have completed very basic CRUD setup for
  Locations. Used Jules to produce the boilerplate for the Update and
  Delete operations, not sure it saved me much time but oh well, it
  was an experiment. Next step is to make all this compile on Cloud,
  since I now have two steps, the proto and then the server.
\item[June 13th] Finally completed migration setup for Cloud SQL using
  \verb|migrate.go| script. Gemini was helpful with the basic
  framework but extremely sloppy about getting the strings correct: It
  tried to set me up with a \verb|cloudsqlconn| custom dialer but gave
  me a \verb|unix| DSN. It insisted that the user for IAM
  authentication must be the full email address,
  \verb|runner@whatever.com|, but did not distinguish that from the
  database user for MySQL, which has a length limit of 32 characters
  and was consequently just \verb|runner|. Very much not oneshot
  success here, although on the other hand I'm not sure I would have
  succeeded at all without getting that basic framework, so. Room for
  improvement but still helpful, really.
\end{itemize}

